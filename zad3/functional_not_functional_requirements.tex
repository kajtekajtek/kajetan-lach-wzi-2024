\documentclass[a4paper, 11pt]{article}
\usepackage{graphicx}

\title{%
\vspace{-2.5cm}
\textbf{Nintendo Entertainment System Emulator}\\
Functional and Non-Functional Requirements
}
\date{\today}
\author{Kajetan Lach}

\begin{document}

\maketitle

\section{Introduction}
This document contains \textbf{functional and not-functional requirements} of our Nintendo Entertainment System emulator project.

\section{Functional Requirements}

\subsection{ROM Loading}
To emulate any program, first step is loading it into memory. The emulator must provide a file selection interface that allows users to load NES ROM files easily. The goal is to successfully load a valid ROM and initiate gameplay.

\subsection{Basic Gameplay Emulation}
The emulator's core functionality is accurate emulation. It must faithfully replicate the NES CPU, PPU and APU to deliver a playable experience for most NES games. This requires correct graphics, sound and input responses for a sample set of popular ROMs.

\subsection{Debugging Tools}
Debugging tools are a key component of the emulator's educational value. The emulator should offer features such as step-by-step execution, memory content inspection and register state visualization. These tools must be clear, intuitive and well-illustrated with graphical representations to enhance usability and understanding 

\subsection{Error Handling}
The emulator must display user-friendly error messages when encountering critical failures. Error messages should be design so that it's easy to understand what happened when program crashed. Easy error reporting functionality should also be a part of appropriate error handling.

\subsection{Settings}
Although advanced configuration is not in scope of this project, our software must allow users to adjust basic quality-of-life settings such as screen scaling, aspect ratio, resolution and audio volume.

\subsection{Input Configuration}
Simple input configuration is another functionality that emulator shall provide for its user for the sake of basic quality-of-life matter. The program must allow users to configure and map keyboard inputs to NES actions and use them to control gameplay in comfortable manner.

\section{Non-Functional Requirements}

\subsection{Performance}

\subsection{Documentation, Code Base \& Other Resources}

\subsection{Usability}

\end{document}