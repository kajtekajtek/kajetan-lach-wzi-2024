\documentclass[a4paper, 11pt]{article}
\usepackage{graphicx}

\title{%
\vspace{-2.5cm}
\textbf{Nintendo Entertainment System Emulator}\\
Functional and Non-Functional Requirements
}
\date{\today}
\author{Kajetan Lach}

\begin{document}

\maketitle

\section{Introduction}
This document contains \textbf{functional and not-functional requirements} of our Nintendo Entertainment System emulator project.

\section{Functional Requirements}

\subsection{ROM Loading}
To emulate any program, first step is loading it into memory. The emulator must provide a file selection interface that allows users to load NES ROM files easily. The goal is to successfully load a valid ROM and initiate gameplay.

\subsection{Basic Gameplay Emulation}
The emulator's core functionality is accurate emulation. It must faithfully replicate the NES CPU, PPU and APU to deliver a playable experience for most NES games. This requires correct graphics, sound and input responses for a sample set of popular ROMs.

\subsection{Debugging Tools}
Debugging tools are a key component of the emulator's educational value. The emulator should offer features such as step-by-step execution, memory content inspection and register state visualization. These tools must be clear, intuitive and well-illustrated with graphical representations to enhance usability and understanding 

\subsection{Error Handling}
The emulator must display user-friendly error messages in the event of critical failures. These messages should clearly explain what went wrong, making it easier for users to understand program crashes. Additionally, the software should include a simple error-reporting feature as part of robust error handling.

\subsection{Settings}
While advanced configuration is outside the scope of this project, the emulator should support basic quality-of-life settings. Users must be able to adjust options such as screen scaling, aspect ratio, resolution and audio volume.

\subsection{Input Configuration}
Input configuration is another essential quality-of-life feature. The emulator must allow users to map keyboard inputs to NES actions, enabling comfortable and customizable gameplay controls.

\section{Non-Functional Requirements}

\subsection{Performance}
The emulator must deliver performance sufficient to run most NES games smnoothly, without noticeable input lag, graphical glitches or issues caused by refresh rate differences. Benchmark testing is essential to ensure stable performance across a range of supported games.
\subsection{Documentation, Code Base \& Other Resources}
Our software must include comprehensive user documentation that covers setup, usage, debugging and troubleshooting. The emulator should also feature a clean, well-organized and detailed codebase. Furthermore, the project site should offer educational resources, including guides, tutorials and videos on emulation and fundamental computer systems concepts.
\subsection{Usability}
The emulator's user interface must be intuitive and accessible, featuring clear labels, straightforward navigation, and easy-to-use debugging options. Usability tests should be conducted to ensure that the software is user-friendly for both beginners and experienced users.

\end{document}